\section{Introducción}

La caracterización de Vessel juega un importante rol en los diagnósticos médicos. Es por ello, que las tareas como la caracterización de su anchor, color, reflectividad, tortuosidad y ramas abnormales son necesarias. El conocimiento de la localización de las lineas de Vessel es de ayuda para la vista de una retinopatía diabética, ya que reduce el número de falsos positivos en la detección de microaneurismas\cite{Spencer_1996,Allan_1998,Larsen_2003}. Cuando el número de ramas o de imágenes es grande, la tarea manual de caracterizar el deliniado de las lineas de Vessel se vuelve un trabajo tedioso y complicado. Con la ayuda de algoritmos de segmentación se ha logrado subsanar la tarea de localizar las lineas de Vessel en una gran cantidad de imágenes. El estudio de esta tarea aumento al incio del milenio, planteando algoritmos basados en detección de bordes\cite{Staal_2004} hasta hoy en día que el estado del arte se encuentra en el uso de aprendizaje profundo\cite{Abber_2022} para clasificar y obtener diagnósticos de las lineas de Vessel.